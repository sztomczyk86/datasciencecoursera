\PassOptionsToPackage{unicode=true}{hyperref} % options for packages loaded elsewhere
\PassOptionsToPackage{hyphens}{url}
%
\documentclass[]{article}
\usepackage{lmodern}
\usepackage{amssymb,amsmath}
\usepackage{ifxetex,ifluatex}
\usepackage{fixltx2e} % provides \textsubscript
\ifnum 0\ifxetex 1\fi\ifluatex 1\fi=0 % if pdftex
  \usepackage[T1]{fontenc}
  \usepackage[utf8]{inputenc}
  \usepackage{textcomp} % provides euro and other symbols
\else % if luatex or xelatex
  \usepackage{unicode-math}
  \defaultfontfeatures{Ligatures=TeX,Scale=MatchLowercase}
\fi
% use upquote if available, for straight quotes in verbatim environments
\IfFileExists{upquote.sty}{\usepackage{upquote}}{}
% use microtype if available
\IfFileExists{microtype.sty}{%
\usepackage[]{microtype}
\UseMicrotypeSet[protrusion]{basicmath} % disable protrusion for tt fonts
}{}
\IfFileExists{parskip.sty}{%
\usepackage{parskip}
}{% else
\setlength{\parindent}{0pt}
\setlength{\parskip}{6pt plus 2pt minus 1pt}
}
\usepackage{hyperref}
\hypersetup{
            pdftitle={Reproducible research Assignment 1 by S. Tomczyk},
            pdfborder={0 0 0},
            breaklinks=true}
\urlstyle{same}  % don't use monospace font for urls
\usepackage[margin=1in]{geometry}
\usepackage{color}
\usepackage{fancyvrb}
\newcommand{\VerbBar}{|}
\newcommand{\VERB}{\Verb[commandchars=\\\{\}]}
\DefineVerbatimEnvironment{Highlighting}{Verbatim}{commandchars=\\\{\}}
% Add ',fontsize=\small' for more characters per line
\usepackage{framed}
\definecolor{shadecolor}{RGB}{248,248,248}
\newenvironment{Shaded}{\begin{snugshade}}{\end{snugshade}}
\newcommand{\AlertTok}[1]{\textcolor[rgb]{0.94,0.16,0.16}{#1}}
\newcommand{\AnnotationTok}[1]{\textcolor[rgb]{0.56,0.35,0.01}{\textbf{\textit{#1}}}}
\newcommand{\AttributeTok}[1]{\textcolor[rgb]{0.77,0.63,0.00}{#1}}
\newcommand{\BaseNTok}[1]{\textcolor[rgb]{0.00,0.00,0.81}{#1}}
\newcommand{\BuiltInTok}[1]{#1}
\newcommand{\CharTok}[1]{\textcolor[rgb]{0.31,0.60,0.02}{#1}}
\newcommand{\CommentTok}[1]{\textcolor[rgb]{0.56,0.35,0.01}{\textit{#1}}}
\newcommand{\CommentVarTok}[1]{\textcolor[rgb]{0.56,0.35,0.01}{\textbf{\textit{#1}}}}
\newcommand{\ConstantTok}[1]{\textcolor[rgb]{0.00,0.00,0.00}{#1}}
\newcommand{\ControlFlowTok}[1]{\textcolor[rgb]{0.13,0.29,0.53}{\textbf{#1}}}
\newcommand{\DataTypeTok}[1]{\textcolor[rgb]{0.13,0.29,0.53}{#1}}
\newcommand{\DecValTok}[1]{\textcolor[rgb]{0.00,0.00,0.81}{#1}}
\newcommand{\DocumentationTok}[1]{\textcolor[rgb]{0.56,0.35,0.01}{\textbf{\textit{#1}}}}
\newcommand{\ErrorTok}[1]{\textcolor[rgb]{0.64,0.00,0.00}{\textbf{#1}}}
\newcommand{\ExtensionTok}[1]{#1}
\newcommand{\FloatTok}[1]{\textcolor[rgb]{0.00,0.00,0.81}{#1}}
\newcommand{\FunctionTok}[1]{\textcolor[rgb]{0.00,0.00,0.00}{#1}}
\newcommand{\ImportTok}[1]{#1}
\newcommand{\InformationTok}[1]{\textcolor[rgb]{0.56,0.35,0.01}{\textbf{\textit{#1}}}}
\newcommand{\KeywordTok}[1]{\textcolor[rgb]{0.13,0.29,0.53}{\textbf{#1}}}
\newcommand{\NormalTok}[1]{#1}
\newcommand{\OperatorTok}[1]{\textcolor[rgb]{0.81,0.36,0.00}{\textbf{#1}}}
\newcommand{\OtherTok}[1]{\textcolor[rgb]{0.56,0.35,0.01}{#1}}
\newcommand{\PreprocessorTok}[1]{\textcolor[rgb]{0.56,0.35,0.01}{\textit{#1}}}
\newcommand{\RegionMarkerTok}[1]{#1}
\newcommand{\SpecialCharTok}[1]{\textcolor[rgb]{0.00,0.00,0.00}{#1}}
\newcommand{\SpecialStringTok}[1]{\textcolor[rgb]{0.31,0.60,0.02}{#1}}
\newcommand{\StringTok}[1]{\textcolor[rgb]{0.31,0.60,0.02}{#1}}
\newcommand{\VariableTok}[1]{\textcolor[rgb]{0.00,0.00,0.00}{#1}}
\newcommand{\VerbatimStringTok}[1]{\textcolor[rgb]{0.31,0.60,0.02}{#1}}
\newcommand{\WarningTok}[1]{\textcolor[rgb]{0.56,0.35,0.01}{\textbf{\textit{#1}}}}
\usepackage{graphicx,grffile}
\makeatletter
\def\maxwidth{\ifdim\Gin@nat@width>\linewidth\linewidth\else\Gin@nat@width\fi}
\def\maxheight{\ifdim\Gin@nat@height>\textheight\textheight\else\Gin@nat@height\fi}
\makeatother
% Scale images if necessary, so that they will not overflow the page
% margins by default, and it is still possible to overwrite the defaults
% using explicit options in \includegraphics[width, height, ...]{}
\setkeys{Gin}{width=\maxwidth,height=\maxheight,keepaspectratio}
\setlength{\emergencystretch}{3em}  % prevent overfull lines
\providecommand{\tightlist}{%
  \setlength{\itemsep}{0pt}\setlength{\parskip}{0pt}}
\setcounter{secnumdepth}{0}
% Redefines (sub)paragraphs to behave more like sections
\ifx\paragraph\undefined\else
\let\oldparagraph\paragraph
\renewcommand{\paragraph}[1]{\oldparagraph{#1}\mbox{}}
\fi
\ifx\subparagraph\undefined\else
\let\oldsubparagraph\subparagraph
\renewcommand{\subparagraph}[1]{\oldsubparagraph{#1}\mbox{}}
\fi

% set default figure placement to htbp
\makeatletter
\def\fps@figure{htbp}
\makeatother


\title{Reproducible research Assignment 1 by S. Tomczyk}
\author{}
\date{\vspace{-2.5em}}

\begin{document}
\maketitle

\hypertarget{load-neccessary-packages}{%
\subsection{Load neccessary packages}\label{load-neccessary-packages}}

\begin{Shaded}
\begin{Highlighting}[]
\KeywordTok{library}\NormalTok{(tidyverse)}
\KeywordTok{library}\NormalTok{(lubridate)}
\KeywordTok{library}\NormalTok{(Hmisc)}
\KeywordTok{library}\NormalTok{(gridExtra)}
\end{Highlighting}
\end{Shaded}

\hypertarget{downloading-and-loading-data-into-r}{%
\subsection{Downloading and loading data into
R}\label{downloading-and-loading-data-into-r}}

The script will check if the input file exist in your workingin
directory and if it does not it will download it. Then the file will be
unzipped and loaded into R

\begin{Shaded}
\begin{Highlighting}[]
\ControlFlowTok{if}\NormalTok{(}\OperatorTok{!}\KeywordTok{file.exists}\NormalTok{(}\StringTok{"activity.csv"}\NormalTok{)) \{}
        
        \KeywordTok{download.file}\NormalTok{(}\StringTok{"https://d396qusza40orc.cloudfront.net/repdata%2Fdata%2Factivity.zip"}\NormalTok{,}
              \StringTok{"data.zip"}\NormalTok{)}
        
        \KeywordTok{unzip}\NormalTok{(}\StringTok{"data.zip"}\NormalTok{)}

\NormalTok{         activity <-}\StringTok{ }\KeywordTok{read.csv}\NormalTok{(}\StringTok{"activity.csv"}\NormalTok{, }\DataTypeTok{header =} \OtherTok{TRUE}\NormalTok{, }
                             \DataTypeTok{colClasses =} \KeywordTok{c}\NormalTok{(}\StringTok{"numeric"}\NormalTok{, }\StringTok{"Date"}\NormalTok{, }\StringTok{"character"}\NormalTok{))}

\NormalTok{\} }\ControlFlowTok{else}\NormalTok{ \{}
        
\NormalTok{        activity <-}\StringTok{ }\KeywordTok{read.csv}\NormalTok{(}\StringTok{"activity.csv"}\NormalTok{, }\DataTypeTok{header =} \OtherTok{TRUE}\NormalTok{, }
                             \DataTypeTok{colClasses =} \KeywordTok{c}\NormalTok{(}\StringTok{"numeric"}\NormalTok{, }\StringTok{"Date"}\NormalTok{, }\StringTok{"character"}\NormalTok{))}
        
\NormalTok{\}}
\end{Highlighting}
\end{Shaded}

\hypertarget{look-at-the-structure-and-summary-of-the-loaded-data}{%
\subsubsection{Look at the structure and summary of the loaded
data}\label{look-at-the-structure-and-summary-of-the-loaded-data}}

\begin{Shaded}
\begin{Highlighting}[]
\KeywordTok{str}\NormalTok{(activity)}
\end{Highlighting}
\end{Shaded}

\begin{verbatim}
## 'data.frame':    17568 obs. of  3 variables:
##  $ steps   : num  NA NA NA NA NA NA NA NA NA NA ...
##  $ date    : Date, format: "2012-10-01" "2012-10-01" ...
##  $ interval: chr  "0" "5" "10" "15" ...
\end{verbatim}

\begin{Shaded}
\begin{Highlighting}[]
\KeywordTok{summary}\NormalTok{(activity)}
\end{Highlighting}
\end{Shaded}

\begin{verbatim}
##      steps             date              interval        
##  Min.   :  0.00   Min.   :2012-10-01   Length:17568      
##  1st Qu.:  0.00   1st Qu.:2012-10-16   Class :character  
##  Median :  0.00   Median :2012-10-31   Mode  :character  
##  Mean   : 37.38   Mean   :2012-10-31                     
##  3rd Qu.: 12.00   3rd Qu.:2012-11-15                     
##  Max.   :806.00   Max.   :2012-11-30                     
##  NA's   :2304
\end{verbatim}

\hypertarget{calculate-the-sum-of-the-steps-taken-each-day}{%
\subsubsection{Calculate the sum of the steps taken each
day}\label{calculate-the-sum-of-the-steps-taken-each-day}}

\begin{Shaded}
\begin{Highlighting}[]
\NormalTok{stepsDaily <-}\StringTok{ }\NormalTok{activity }\OperatorTok\StringTok{ }\KeywordTok{group_by}\NormalTok{(date) }\OperatorTok\StringTok{ }\KeywordTok{summarise}\NormalTok{(}\DataTypeTok{sum =} \KeywordTok{sum}\NormalTok{(steps))}
\end{Highlighting}
\end{Shaded}

\begin{verbatim}
## `summarise()` ungrouping output (override with `.groups` argument)
\end{verbatim}

\begin{Shaded}
\begin{Highlighting}[]
\KeywordTok{head}\NormalTok{(stepsDaily)}
\end{Highlighting}
\end{Shaded}

\begin{verbatim}
## # A tibble: 6 x 2
##   date         sum
##   <date>     <dbl>
## 1 2012-10-01    NA
## 2 2012-10-02   126
## 3 2012-10-03 11352
## 4 2012-10-04 12116
## 5 2012-10-05 13294
## 6 2012-10-06 15420
\end{verbatim}

Produce histogram of the total number of steps taken per day

\begin{Shaded}
\begin{Highlighting}[]
\KeywordTok{ggplot}\NormalTok{(}\DataTypeTok{data =}\NormalTok{ stepsDaily, }\KeywordTok{aes}\NormalTok{(sum)) }\OperatorTok{+}\StringTok{ }
\StringTok{        }\KeywordTok{geom_histogram}\NormalTok{(}\DataTypeTok{binwidth =} \KeywordTok{range}\NormalTok{(stepsDaily}\OperatorTok{$}\NormalTok{sum, }\DataTypeTok{na.rm =} \OtherTok{TRUE}\NormalTok{)[}\DecValTok{2}\NormalTok{]}\OperatorTok{/}\DecValTok{30}\NormalTok{) }\OperatorTok{+}
\StringTok{        }\KeywordTok{xlab}\NormalTok{(}\StringTok{"Total number of steps"}\NormalTok{) }\OperatorTok{+}\StringTok{ }\KeywordTok{ylab}\NormalTok{(}\StringTok{"Number of Days"}\NormalTok{)}\OperatorTok{+}
\StringTok{        }\KeywordTok{labs}\NormalTok{(}\DataTypeTok{title=}\StringTok{"Hisogram of the total number of steps per day"}\NormalTok{)}
\end{Highlighting}
\end{Shaded}

\includegraphics{RR_Assignment1_files/figure-latex/unnamed-chunk-5-1.pdf}

\hypertarget{calculate-mean-and-median-number-od-steps-taken-each-day.}{%
\subsubsection{Calculate mean and median number od steps taken each
day.}\label{calculate-mean-and-median-number-od-steps-taken-each-day.}}

\begin{Shaded}
\begin{Highlighting}[]
\NormalTok{stepsM  <-activity }\OperatorTok\StringTok{ }\KeywordTok{group_by}\NormalTok{(date) }\OperatorTok\StringTok{ }
\StringTok{        }\KeywordTok{summarise}\NormalTok{(}\DataTypeTok{mean =} \KeywordTok{mean}\NormalTok{(steps, }\DataTypeTok{na.rm =} \OtherTok{TRUE}\NormalTok{), }
                  \DataTypeTok{median =} \KeywordTok{median}\NormalTok{(steps, }\DataTypeTok{na.rm =} \OtherTok{TRUE}\NormalTok{))}
\end{Highlighting}
\end{Shaded}

\begin{verbatim}
## `summarise()` ungrouping output (override with `.groups` argument)
\end{verbatim}

\begin{Shaded}
\begin{Highlighting}[]
\KeywordTok{print}\NormalTok{(stepsM)}
\end{Highlighting}
\end{Shaded}

\begin{verbatim}
## # A tibble: 61 x 3
##    date          mean median
##    <date>       <dbl>  <dbl>
##  1 2012-10-01 NaN         NA
##  2 2012-10-02   0.438      0
##  3 2012-10-03  39.4        0
##  4 2012-10-04  42.1        0
##  5 2012-10-05  46.2        0
##  6 2012-10-06  53.5        0
##  7 2012-10-07  38.2        0
##  8 2012-10-08 NaN         NA
##  9 2012-10-09  44.5        0
## 10 2012-10-10  34.4        0
## # ... with 51 more rows
\end{verbatim}

\hypertarget{plot-the-avergage-number-of-steps-taken-each-day}{%
\subsubsection{Plot the avergage number of steps taken each
day}\label{plot-the-avergage-number-of-steps-taken-each-day}}

\begin{Shaded}
\begin{Highlighting}[]
\KeywordTok{ggplot}\NormalTok{(}\DataTypeTok{data =}\NormalTok{ stepsM, }\KeywordTok{aes}\NormalTok{(date, mean))}\OperatorTok{+}
\StringTok{        }\KeywordTok{geom_point}\NormalTok{() }\OperatorTok{+}\StringTok{ }\KeywordTok{geom_smooth}\NormalTok{() }\OperatorTok{+}
\StringTok{        }\KeywordTok{labs}\NormalTok{(}\DataTypeTok{title =} \StringTok{"Average number of steps per day"}\NormalTok{) }\OperatorTok{+}
\StringTok{        }\KeywordTok{xlab}\NormalTok{(}\StringTok{"Day"}\NormalTok{) }\OperatorTok{+}
\StringTok{        }\KeywordTok{ylab}\NormalTok{(}\StringTok{"Average number of steps"}\NormalTok{)}
\end{Highlighting}
\end{Shaded}

\begin{verbatim}
## `geom_smooth()` using method = 'loess' and formula 'y ~ x'
\end{verbatim}

\includegraphics{RR_Assignment1_files/figure-latex/unnamed-chunk-7-1.pdf}

\hypertarget{find-the-5-minute-interval-that-on-average-had-maximum-number-of-steps}{%
\subsubsection{Find the 5 minute interval that on average had maximum
number of
steps}\label{find-the-5-minute-interval-that-on-average-had-maximum-number-of-steps}}

\begin{Shaded}
\begin{Highlighting}[]
\NormalTok{stepsMax <-}\StringTok{ }\NormalTok{activity }\OperatorTok\StringTok{ }\KeywordTok{group_by}\NormalTok{(interval) }\OperatorTok\StringTok{ }
\StringTok{        }\KeywordTok{summarise}\NormalTok{(}\DataTypeTok{mean =} \KeywordTok{mean}\NormalTok{(steps, }\DataTypeTok{na.rm =}\NormalTok{ T)) }\OperatorTok
\StringTok{        }\KeywordTok{filter}\NormalTok{(mean}\OperatorTok{==}\KeywordTok{max}\NormalTok{(mean)) }\OperatorTok\StringTok{ }\KeywordTok{print}\NormalTok{()}
\end{Highlighting}
\end{Shaded}

\begin{verbatim}
## `summarise()` ungrouping output (override with `.groups` argument)
\end{verbatim}

\begin{verbatim}
## # A tibble: 1 x 2
##   interval  mean
##   <chr>    <dbl>
## 1 835       206.
\end{verbatim}

The 5 minute interval with highest average number of steps was 8.35 in
the morning

\hypertarget{impute-the-missing-data-based-used-random-values-from-the-range-of-recorded-steps}{%
\subsubsection{Impute the missing data based used random values from the
range of recorded
steps}\label{impute-the-missing-data-based-used-random-values-from-the-range-of-recorded-steps}}

\begin{Shaded}
\begin{Highlighting}[]
\KeywordTok{set.seed}\NormalTok{(}\DecValTok{666}\NormalTok{)}
\NormalTok{activity}\OperatorTok{$}\NormalTok{imputed_steps <-}\StringTok{ }\KeywordTok{impute}\NormalTok{(activity}\OperatorTok{$}\NormalTok{steps, }\StringTok{'random'}\NormalTok{)}

\KeywordTok{head}\NormalTok{(activity)}
\end{Highlighting}
\end{Shaded}

\begin{verbatim}
##   steps       date interval imputed_steps
## 1    NA 2012-10-01        0             0
## 2    NA 2012-10-01        5             0
## 3    NA 2012-10-01       10             0
## 4    NA 2012-10-01       15           250
## 5    NA 2012-10-01       20             0
## 6    NA 2012-10-01       25             0
\end{verbatim}

Calculate total number of steps per day for the imputed data

\begin{Shaded}
\begin{Highlighting}[]
\NormalTok{stepsDaily.imp <-}\StringTok{ }\NormalTok{activity }\OperatorTok\StringTok{ }\KeywordTok{group_by}\NormalTok{(date) }\OperatorTok\StringTok{ }\KeywordTok{summarise}\NormalTok{(}\DataTypeTok{sum =} \KeywordTok{sum}\NormalTok{(imputed_steps))}
\end{Highlighting}
\end{Shaded}

\begin{verbatim}
## `summarise()` ungrouping output (override with `.groups` argument)
\end{verbatim}

\begin{Shaded}
\begin{Highlighting}[]
\KeywordTok{head}\NormalTok{(stepsDaily.imp)}
\end{Highlighting}
\end{Shaded}

\begin{verbatim}
## # A tibble: 6 x 2
##   date         sum
##   <date>     <dbl>
## 1 2012-10-01 10355
## 2 2012-10-02   126
## 3 2012-10-03 11352
## 4 2012-10-04 12116
## 5 2012-10-05 13294
## 6 2012-10-06 15420
\end{verbatim}

Compare the histograms of total number of steps with or without data
imputation

\begin{Shaded}
\begin{Highlighting}[]
\NormalTok{hist <-}\StringTok{ }\KeywordTok{ggplot}\NormalTok{(}\DataTypeTok{data =}\NormalTok{ stepsDaily, }\KeywordTok{aes}\NormalTok{(sum)) }\OperatorTok{+}\StringTok{ }
\StringTok{        }\KeywordTok{geom_histogram}\NormalTok{(}\DataTypeTok{binwidth =} \KeywordTok{range}\NormalTok{(stepsDaily}\OperatorTok{$}\NormalTok{sum, }\DataTypeTok{na.rm =} \OtherTok{TRUE}\NormalTok{)[}\DecValTok{2}\NormalTok{]}\OperatorTok{/}\DecValTok{30}\NormalTok{) }\OperatorTok{+}
\StringTok{        }\KeywordTok{xlab}\NormalTok{(}\StringTok{"Total number of steps"}\NormalTok{) }\OperatorTok{+}\StringTok{ }\KeywordTok{ylab}\NormalTok{(}\StringTok{"Number of Days"}\NormalTok{)}\OperatorTok{+}
\StringTok{        }\KeywordTok{labs}\NormalTok{(}\DataTypeTok{title=}\StringTok{"Hisogram of the total number of steps per day"}\NormalTok{)}\OperatorTok{+}
\StringTok{        }\KeywordTok{ylim}\NormalTok{(}\DecValTok{0}\NormalTok{, }\DecValTok{10}\NormalTok{)}
        

\NormalTok{hist.imp <-}\StringTok{ }\KeywordTok{ggplot}\NormalTok{(}\DataTypeTok{data =}\NormalTok{ stepsDaily.imp, }\KeywordTok{aes}\NormalTok{(sum)) }\OperatorTok{+}\StringTok{ }
\StringTok{        }\KeywordTok{geom_histogram}\NormalTok{(}\DataTypeTok{binwidth =} \KeywordTok{range}\NormalTok{(stepsDaily}\OperatorTok{$}\NormalTok{sum, }\DataTypeTok{na.rm =} \OtherTok{TRUE}\NormalTok{)[}\DecValTok{2}\NormalTok{]}\OperatorTok{/}\DecValTok{30}\NormalTok{) }\OperatorTok{+}
\StringTok{        }\KeywordTok{xlab}\NormalTok{(}\StringTok{"Total number of steps"}\NormalTok{) }\OperatorTok{+}\StringTok{ }\KeywordTok{ylab}\NormalTok{(}\StringTok{"Number of Days"}\NormalTok{)}\OperatorTok{+}
\StringTok{        }\KeywordTok{labs}\NormalTok{(}\DataTypeTok{title=}\StringTok{"Hisogram of the total number of steps per day with imputed data"}\NormalTok{) }\OperatorTok{+}
\StringTok{        }\KeywordTok{ylim}\NormalTok{(}\DecValTok{0}\NormalTok{, }\DecValTok{10}\NormalTok{)}

\KeywordTok{grid.arrange}\NormalTok{(hist, hist.imp, }\DataTypeTok{nrow=}\DecValTok{2}\NormalTok{)}
\end{Highlighting}
\end{Shaded}

\includegraphics{RR_Assignment1_files/figure-latex/unnamed-chunk-11-1.pdf}

\hypertarget{plot-the-average-number-of-steps-taken-per-5-minute-interval-across-weekdays-and-weekends}{%
\subsubsection{Plot the average number of steps taken per 5-minute
interval across weekdays and
weekends}\label{plot-the-average-number-of-steps-taken-per-5-minute-interval-across-weekdays-and-weekends}}

Add the WeekEnd WeekDay labels to the days using lubridate package

\begin{Shaded}
\begin{Highlighting}[]
\NormalTok{activity}\OperatorTok{$}\NormalTok{weekday <-}\StringTok{ }\KeywordTok{wday}\NormalTok{(activity}\OperatorTok{$}\NormalTok{date, }\DataTypeTok{label =}\NormalTok{ T, }\DataTypeTok{abbr =}\NormalTok{ F)}

\NormalTok{stepsWD <-}\StringTok{ }\NormalTok{activity }\OperatorTok\StringTok{ }\KeywordTok{mutate}\NormalTok{(}\DataTypeTok{wd =} \KeywordTok{if_else}\NormalTok{(weekday }\OperatorTok\StringTok{ }
\StringTok{                                }\KeywordTok{c}\NormalTok{(}\StringTok{"Samstag"}\NormalTok{, }\StringTok{"Sonntag"}\NormalTok{), }\StringTok{"WeekEnd"}\NormalTok{, }\StringTok{"WeekDay"}\NormalTok{)) }\OperatorTok
\StringTok{                        }\KeywordTok{group_by}\NormalTok{(interval, wd) }\OperatorTok\StringTok{ }
\StringTok{                        }\KeywordTok{summarise}\NormalTok{(}\DataTypeTok{mean =} \KeywordTok{mean}\NormalTok{(steps, }\DataTypeTok{na.rm =} \OtherTok{TRUE}\NormalTok{))}
\end{Highlighting}
\end{Shaded}

\begin{verbatim}
## `summarise()` regrouping output by 'interval' (override with `.groups` argument)
\end{verbatim}

\begin{Shaded}
\begin{Highlighting}[]
\NormalTok{stepsWD}\OperatorTok{$}\NormalTok{interval <-}\StringTok{ }\KeywordTok{as.numeric}\NormalTok{(stepsWD}\OperatorTok{$}\NormalTok{interval)}
\end{Highlighting}
\end{Shaded}

Produce the plot showing the difference in number of steps between
weekend and weekdays.

\begin{Shaded}
\begin{Highlighting}[]
\KeywordTok{ggplot}\NormalTok{(stepsWD, }\KeywordTok{aes}\NormalTok{(interval, mean)) }\OperatorTok{+}
\StringTok{        }\KeywordTok{geom_point}\NormalTok{(}\KeywordTok{aes}\NormalTok{(}\DataTypeTok{colour =}\NormalTok{ wd)) }\OperatorTok{+}
\StringTok{        }\KeywordTok{ylab}\NormalTok{(}\StringTok{"Average number of steps"}\NormalTok{)}\OperatorTok{+}
\StringTok{        }\KeywordTok{xlab}\NormalTok{(}\StringTok{"5-minutes intervals"}\NormalTok{)}\OperatorTok{+}
\StringTok{        }\KeywordTok{labs}\NormalTok{(}\DataTypeTok{title =} \StringTok{"Comparison between number of steps in Weekend vs Weekday"}\NormalTok{)}\OperatorTok{+}
\StringTok{        }\KeywordTok{facet_wrap}\NormalTok{(}\OperatorTok{~}\NormalTok{wd, }\DataTypeTok{ncol=}\DecValTok{2}\NormalTok{)}
\end{Highlighting}
\end{Shaded}

\includegraphics{RR_Assignment1_files/figure-latex/unnamed-chunk-13-1.pdf}

\end{document}
